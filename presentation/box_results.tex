Each part (route between two consequent stations) of the path of the train contains different number of routes that goes through this part. For example, if the path has 15 stations the number of routes going through part between the first two stations is 14 and this number in the middle of the way is 56. Therefore, the prices should be higher (even if the demand is the same) in the middle routes then the edge ones. This idea is supported by numerical result with six stations. The table shows estimated optimum ticket prices for pairs of boarding (rows) and exiting (columns) stations.

\begin{center}
			\begin{tabular}{cccccc}
				\hline
				 & 2 & 3 & 4 & 5 & 6 \\ 
				\hline
				1 & 171.0 & 215.8 & 336.2 & 441.1 & 487.2 \\ 
				2 &  & 190.8 & 305.1 & 385.2 & 448.8 \\ 
				3 &  &  & 253.0 & 310.8 & 342.9 \\ 
				4 &  &  &  & 137.1 & 179.6 \\ 
				5 &  &  &  &  & 114.9 \\ 
				 \hline
			\end{tabular}    
\end{center}