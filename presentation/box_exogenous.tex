It is possible to separate the dependence on price from the intensity of the Poisson process (see Proposition 1). With this we can simulate the demand for single route as compound Poisson process where intensity of arrivals of passengers (passenger show interest to buy a ticket) depends only on time and the alternative distribution (indicating whether the passenger actually buys the ticket) depends on both time and price. We can simulate for each passenger the maximum price that is willing to pay. Once we want to evaluate the outcome for different price we distinguish passengers that buy the ticket or not. Therefore, we do not need to simulate the demand for each price separately. \\

\textbf{Proposition 1.}
\emph{
		Let $N = \{N_t, t \geq 0\}$ be an inhomogeneous Poisson process with intensity $\lambda(t)$ and jump times $T_1, T_2, ...$. Let $\pi: \Rplus \to (0,1)$ be a right-continuous function and $Z_i \sim Alt(\pi(T_i))$. Then the family of random variables
		\[
			M_t = \sum_{i=1}^{\infty} Z_i \times \mathbb{I} [t \geq T_i].
		\]
		is an inhomogeneous Poisson process with intensity $\lambda(t) \pi(t)$.
}
