\pagestyle{empty}
\hypersetup{pageanchor=false}
\begin{center}

\centerline{\mbox{\includegraphics[width=166mm]{figures/mfflogo_new.pdf}}} % 166mm

\vspace{-8mm}
\vfill

{\bf\Large MASTER THESIS}

\vfill

{\LARGE Jan Kislinger}

\vspace{15mm}

{\LARGE\bfseries Dynamic Fare Model}

\vfill

Department of Probability and Mathematical Statistics

\vfill

\begin{tabular}{rl}

Supervisor of the master thesis: & Doc. RNDr. Petr Lachout, CSc. \\
\noalign{\vspace{2mm}}
Study programme: & Mathematics \\
\noalign{\vspace{2mm}}
Study branch: & 
	\begin{minipage}[t]{220px}%
		Probability, Mathematical Statistics and Econometrics
	\end{minipage} \tabularnewline
\end{tabular}

\vfill

% Zde doplňte rok
Prague 2017

\end{center}


%%% At this place, the printed version includes a page containing the
%%% photocopy of the official signed "Master thesis assignment".
%%% This should not be included in the electronic version. 



\newpage
%%% Page containing a legal statement
\vspace*{\stretch{8}}

\noindent
I declare that I carried out this master thesis independently, and only with the cited
sources, literature and other professional sources.

\medskip\noindent
I understand that my work relates to the rights and obligations under the Act No.
121/2000 Coll., the Copyright Act, as amended, in particular the fact that the Charles
University in Prague has the right to conclude a license agreement on the use of this
work as a school work pursuant to Section 60 paragraph 1 of the Copyright Act.

\vspace{18mm}
\noindent
In Prague on January 4, 2017%\today
\hspace*{\fill}
\hspace*{\fill}
\hspace*{\fill}
Jan Kislinger
\hspace*{\fill}

\vspace*{\stretch{1}}



\newpage
%%% Czech and English abstracts

\vbox to 0.5\vsize{
\setlength\parindent{0mm}
\setlength\parskip{5mm}

N\'azev pr\'ace:
Dynamick\'y model ceny j\'izdn\'eho

Autor:
Bc. Jan Kislinger

Katedra:  
Katedra pravd\v{e}podobnosti a~matematick\'e statistiky

Vedouc\'\i\ bakal\'a\v{r}sk\'e pr\'ace:
Doc. RNDr. Petr Lachout, CSc., Katedra prav\-d\v{e}\-po\-dob\-nos\-ti a~matematick\'e statistiky

Abstrakt:
Problém hledání dynamického modelu pro ceny jízdného se skládá ze dvou úloh -- odhadování poptávky po vlakových jízdenkách a vícestupňová optimalizace ceny jízdného.
V této práci představujeme model nehomogenního markovského řetězce, který používáme pro vývoj prodeje jízdenek.
Z důvodu velikosti stavového prostoru je nutné řešit optimalizační úlohu pomocí simulované optimalizace.
Řešení jednostupňového a dvoustupňového problému je implementováno v jazyce R.
Před samotným praktickým problémem shrnujeme teorii nehomogenních markovských řetězců, kde se podrobněji zaměřujeme na procesy se separovatelnou nehomogenitou.
Dále navrhujeme metody odhadování intenzity markovského procesu založené na teorii maximální věrohodnosti.
Také popisujeme a srovnáváme dva algoritmy simulované optimalizace.

Kl\'{\i}\v{c}ov\'a slova:
nehomogenní markovský proces, odhadování intenzity, simulovaná optimalizace, dynamické jízdné

\vss}

\nobreak\vbox to 0.49\vsize{
\setlength\parindent{0mm}
\setlength\parskip{5mm}

Title:
Dynamic Fare Model

Author:
Bc. Jan Kislinger

Department:
Department of Probability and Mathematical Statistics


Supervisor of the master thesis:
Doc. RNDr. Petr Lachout, CSc., Department of Probability and Mathematical Statistics

Abstract:
The problem of creating dynamic fare model consists of two tasks~-- estimating demand for train tickets and multistage optimization of price of fare.
We introduce a model of inhomogeneous Markov process for the process of selling the tickets in this thesis.
Because of the complexity of the state space the optimization problem needs to be solved using simulation methods.
The solution was implemented in R language for single-stage and two-stage problems.
Before this application we summarize the theory of inhomogeneous Markov process with special attention to process with separable inhomogeneity.
Then we propose methods for estimating the intensity using maximum likelihood theory.
We also describe and compare two algorithms for simulated optimization.

Keywords:
inhomogeneous Markov process, estimating intensity, simulated optimization, dynamic fare
\vss}

%%% A page containing the automatically generated contents of the
%%% master thesis. For a mathematical thesis it is allowed to
%%% place the list of tables and abbreviations at the beginning
%%% of the thesis instead of at the end.

%%% Acknowledgments
\newpage
\openright

\noindent
I would like to thank doc.RNDr. Lachout Petr, CSc., my supervisor of this thesis, for leading me in the right direction.
I also thank to Prof. RNDr. Antoch Jaromír, CSc., who helped me understand and use simulation methods in this thesis.

\newpage
\openright

\pagestyle{plain}
\setcounter{page}{1}

